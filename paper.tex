\documentclass[conference]{IEEEtran}
\IEEEoverridecommandlockouts
\usepackage{cite}
\usepackage{amsmath,amssymb,amsfonts}
\usepackage{algorithmic}
\usepackage{graphicx}
\usepackage{textcomp}
\usepackage{xcolor}
\def\BibTeX{{\rm B\kern-.05em{\sc i\kern-.025em b}\kern-.08em
		T\kern-.1667em\lower.7ex\hbox{E}\kern-.125emX}}
\begin{document}
	
	\title{Integration of 75MW Solar PV Plant: Transmission System Design Analysis}
	
	\author{\IEEEauthorblockN{Brent Dickinson}
		\IEEEauthorblockN{Tianci}
	}
	
	\maketitle
	
\begin{abstract}
	This report presents the design and analysis of transmission system modifications required to integrate a new 75MW solar PV plant while addressing existing system reliability concerns. The study evaluates various transmission line and transformer options to determine the most cost-effective solution that maintains system stability under both normal and N-1 contingency conditions.
\end{abstract}

\section{Introduction}
The integration of renewable energy sources into existing power grids presents technical challenges. We analyze the design requirements and potential solutions for integrating a new 75 MW utility-scale solar photo-voltaic (PV) facility into an existing 37-bus power system while also solving current reliability issues.

We aim to find the most cost-effective transmission system additions required to integrate 75 MW of solar generation at the NEWSOLAR substation. That goal is subordinate to the constraints of ensuring system reliability through dual transmission paths to the facility and resolving existing system violations identified using PowerWorld's contingency analysis.

The design must satisfy the following constraints. Bus voltages must be maintained between 0.95 and 1.10 per unit, while keeping all line flows below 100\% of their thermal limits. The system must maintain stability under both normal operation and N-1 contingency conditions. Additionally, redundant transmission paths to the NEWSOLAR substation are required. The system must accommodate the solar plant operating at both full capacity (75 MW) and offline conditions.

Our approach incorporates contingency analysis, evaluation of available transmission corridors and voltage levels, and assessment of various conductor types and their associated costs. We consider transformer options and substation modifications. We conduct an economic analysis incorporating both capital costs and five-year loss reduction benefits.

Initial contingency analysis reveals specific reliability concerns in the OAK69-BUCKEYE69-APPLE69 corridor, which must be addressed alongside the integration of the new solar facility. The following sections detail the technical analysis, proposed solutions, and economic justification for the recommended design.

\section{Initial System Analysis}
\subsection{Base Case Evaluation}
The existing 37-bus system operates with a total load of 826.3 MW and 275.5 Mvar, served by ten generators producing 837.7 MW. System losses are approximately 10.7 MW, representing 1.3\% of total generation, which indicates reasonably efficient power delivery under normal conditions. The system maintains adequate reactive power compensation through nine switched shunts, collectively providing -122.5 Mvar of reactive support.

PowerWorld's case summary for the existing system is shown in Figure \ref{fig:casesummaryexisting}
\begin{figure}[tbph]
	\centering
	\includegraphics[width=1\linewidth]{figures/case_summary_existing}
	\caption{Case summary for existing system}
	\label{fig:casesummaryexisting}
\end{figure}
\subsection{Contingency Analysis}
PowerWorld's contingency analysis examines conditions when each element of the power system is taken offline. It reveals vulnerability in the 69 kV network in three contingency violations. Specifically, loss of either the PINE138 transformer or the PINE69-APPLE69 line results in overload, with the OAK69-BUCKEYE69 line experiencing loading up to 110.8\% of its thermal limit. These violations suggest that the existing infrastructure is approaching its capacity limits. 

These results are summarized in Table \ref{tab:violations}. A zoomed-in view of the affected areas of the power system is shown in Figure \ref{fig:baseviolations}.
\begin{table}[htbp]
	\caption{Line violations in contingency analysis}
	\begin{center}
		\begin{tabular}{|l|c|c|c|}
			\hline
			\textbf{Contingency} & \textbf{Flow(A)} & \textbf{Limit(A)} & \textbf{\%} \\
			\hline
			\textit{PINE138-PINE69 Xfmr:} & & & \\
			OAK69-BUCKEYE69 & 760.3 & 686.1 & 110.8 \\
			BUCKEYE69-APPLE69 & 454.2 & 418.4 & 108.6 \\
			\hline
			\textit{PINE69-APPLE69 Line:} & & & \\
			OAK69-BUCKEYE69 & 699.2 & 686.1 & 101.9 \\
			\hline
		\end{tabular}
		\label{tab:violations}
	\end{center}
\end{table}
\begin{figure}[tbph]
	\centering
	\includegraphics[width=1\linewidth]{figures/base_violations}
	\caption{Existing contingency violations - problem area}
	\label{fig:baseviolations}
\end{figure}
\section{Design}
The design solution must incorporate new transmission infrastructure to connect the 75 MW solar facility at NEWSOLAR while simultaneously addressing existing problems. There must be redundant transmission paths to NEWSOLAR for reliability, meaning at least two separate transmission lines must be constructed. These lines can be either 69 kV or 138 kV, with NEWSOLAR and all existing substations capable of accommodating either voltage level through installation of transformers. The system must maintain all bus voltages between 0.95 and 1.10 per unit and keep all line flows below their thermal limits under both normal operation and any single-contingency (N-1) scenario.

The optimal solution will minimize total cost, calculated as the construction costs minus any savings from reduced system losses over a 5-year period at \$60/MWh. Construction costs include both fixed components (\$1.25M for 138 kV or \$750k for 69 kV lines) and variable costs based on distance and conductor type. If 138 kV is utilized, additional costs include substation upgrades (\$900k per substation) and necessary transformers (\$1.5M for 101 MVA or \$1.8M for 168 MVA). The solution must resolve existing contingency violations, particularly the overloads observed in the OAK69-BUCKEYE69-APPLE69 corridor, while ensuring reliable operation both when the solar plant is at full output and when it is offline.
\subsection{Transmission Line Parameters}
A note on how to input transmission line parameters into PowerWorld is in order. Table A.4 in the text gives resistance in ohms per conductor per mile for 50$^\circ$C at 60Hz as well as series inductance (ohms/mile/conductor for 1 foot spacing) and shunt capacitance (S/mile/conductor for 1 foot spacing). Resistance pulled from Table A.4 can be used directly in the "Calculate Impedances" dialog box in PowerWorld. But for 69kV lines, the three phases are spaced 2m apart and 4m apart for 138kV lines. We need a way to convert the 1-foot spacing parameters appropriately. Assuming equilateral spacing of lines, Equation 4.5.9 in the text suggests a simple way to adjust series per-mile impedance from the table:
\begin{flalign}
	X_{2m} &= X_{1ft}\cdot\frac{\ln\left(\frac{2\times3.28}{\text{GMR}}\right)}{\ln\left(\frac{1}{\text{GMR}}\right)}
\end{flalign}
where GMR comes from the same table for the conductor of interest. 

We calculate shunt susceptance similarly given the 1-foot spacing shunt reactance and outer diameter given in the table:
\begin{flalign}
	B_{2m} & = \frac{\ln(1/r)}{Xc_{1ft}\cdot\ln(2\times3.28/r)}
\end{flalign}
where $r$ is the conductor's outer radius. Though these are stranded conductors, we treat them as solid since the associated error is small. Note that we multiply this value by 3 for 3 phases. 

Table \ref{tab:tl_params} shows parameters for the 69kV lines which go into PowerWorld's Impedance Calculator. We do not include the 138kV lines for reasons explained in the next section.
\begin{table}
	\begin{tabular}{|l|c|c|c|}
		\hline
		Parameter & Rook & Crow & Condor \\
		\hline
		R ($\Omega$/mile) & 0.1688 & 0.1482 & 0.1378 \\
		X ($\Omega$/mile) & 0.6421 & 0.6352 & 0.6326 \\
		B (S/mile) & $6.63\times10^{-6}$ & $6.71\times10^{-6}$ & $6.78\times10^{-6}$ \\
		\hline
	\end{tabular}
	\vspace{0.5em}
	\caption{Transmission Line Parameters at 60 Hz, 2m Spacing}
	\label{tab:tl_params}
\end{table}
\subsection{Candidate Solutions}
Table \ref{tab:all_conn} shows all possible NEWSOLAR connections to the network, given the provided set of right-of-way paths available to us. The cost of 138kV lines and transformers suggests we can eliminate any solution involving those lines; we will connect NEWSOLAR exclusively with 69kV lines unless we are unable to solve potential contingency violations. 

The approach here is to start with the solution involving least total line distance (case 1). We move to solutions involving longer connections only if we run into contingency violations. Further, we use the least expensive conductors to start and upgrade only if needed. 
\begin{table}[h]
	\centering
	\begin{tabular}{|c|l|c|r|}
		\hline
		\textbf{Case} & \textbf{Destination} & \textbf{Total} & \textbf{Total Cost} \\
		\textbf{number} & \textbf{buses} & \textbf{distance (km)} & \textbf{(M\$)} \\
		\hline
		1 & BK69, AP69 & 12 & 5.94 \\
		2 & BK69, OAK69 & 19 & 8.53 \\
		3 & BK69, OAK138 & 19 & 15.17 \\
		4 & AP69, OAK69 & 19 & 8.53 \\
		5 & AP69, OAK138 & 19 & 15.17 \\
		6 & BK69, PIINE69 & 20 & 8.90 \\
		7 & BK69, PINE138 & 20 & 15.80 \\
		8 & AP69, PINE69 & 20 & 8.90 \\
		9 & AP69, PINE138 & 20 & 15.80 \\
		10 & BK69, MP69 & 21 & 9.27 \\
		11 & AP69, MP69 & 21 & 9.27 \\
		12 & OAK69, PINE69 & 27 & 11.49 \\
		13 & OAK69, PINE138 & 27 & 19.56 \\
		14 & OAK138, PINE69 & 27 & 19.56 \\
		15 & OAK138, PINE138 & 27 & 19.56 \\
		16 & OAK69, MP69 & 28 & 11.86 \\
		17 & OAK138, MP69 & 28 & 20.19 \\
		18 & PINE69, MP69 & 29 & 12.23 \\
		19 & PINE138, MP69 & 29 & 20.82 \\
		\hline
	\end{tabular}
	\vspace{0.5em}
	\caption{Possible NEWSOLAR Connections with Line Costs (M\$)}
	\label{tab:all_conn}
\end{table}

We found that simply installing connections to BUCKEYE and APPLE results in several contingency violations. These violations came from over-currents on the OAK-BUCKEYE line we so we replaced that line with Crow conductors to ensure ample allowance for current. With the replaced line we saw no violations. The total cost for case 1 with this upgrade is shown in Table \ref{tab:cost_breakdown_case1}.
\begin{table}[h!]
	\centering
	\begin{tabular}{|l|r|}
		\hline
		\textbf{Modification} & \textbf{Cost (M\$)} \\ \hline
		\multicolumn{2}{|l|}{\textbf{Replace OAK-BUCKEYE (69kV Crow)}} \\ 
		\hspace{1em} Fixed Cost & 0.75 \\ 
		\hspace{1em} Variable Cost (\$390,000/km for 8 km) & 3.12 \\ 
		\hspace{1em} \textbf{Total} & \textbf{3.87} \\ \hline
		\multicolumn{2}{|l|}{\textbf{New Line NEWSOLAR-BUCKEYE (69kV Rook)}} \\ 
		\hspace{1em} Fixed Cost & 0.75 \\ 
		\hspace{1em} Variable Cost (\$370,000/km for 6 km) & 2.22 \\ 
		\hspace{1em} \textbf{Total} & \textbf{2.97} \\ \hline
		\multicolumn{2}{|l|}{\textbf{New Line NEWSOLAR-APPLE (69kV Rook)}} \\ 
		\hspace{1em} Fixed Cost & 0.75 \\ 
		\hspace{1em} Variable Cost (\$370,000/km for 6 km) & 2.22 \\ 
		\hspace{1em} \textbf{Total} & \textbf{2.97} \\ \hline
		\textbf{Grand Total} & \textbf{9.81} \\ \hline
	\end{tabular}
	\vspace{0.5em}
	\caption{Cost Breakdown for Case 1 Implementation with OAK-BUCKEYE Upgrade}
	\label{tab:cost_breakdown_case1}
\end{table}

This cost exceeds that of cases 2, 4, 6, 8, 10 and 11 as seen in Table \ref{tab:all_conn}. However, the cheapest 6km installation costs \$3M, so for any of these other cases to be preferred there must be no contingency violations. If there are, any corrective line replacement will make that option more expensive than case 1 with upgrade.


\section{Conclusion}
Summary of key findings and recommendations
\end{document}